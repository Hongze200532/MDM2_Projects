% MDM2 LaTeX template
% 2020-11-16
% C L Hall, using some existing material from S J Hogan and R E Wilson

% README: Look at the preamble (the things before \begin{document}). Probably, the only things you will need to change are the bits in "Title, authors, date" immediately below.

% Document class: Unless you are writing a dissertation, "article" is almost always the correct document class. 
\documentclass[11pt,a4paper,twoside]{article}
\usepackage[utf8]{inputenc} % Enables direct typing of special characters

% Title, authors, date
% NOTE: I like to create new commands for theTitle (which appears on the cover page), theShortTitle, and theAuthors that I can call later in the document; you don't need to do this.
\newcommand{\theShortTitle}{Short title of the group project} 
\newcommand{\theTitle}{{\large MDM2 Group Project 2, Group 10} \\ \theShortTitle}
\newcommand{\theAuthors}{Alfie Rootham, Hongze Lin, Qusai Gazzaz, William Bolton, Zach Ball}
\title{\theTitle}
\author{\theAuthors}
\date{\today}

% Page style
% NOTE: The geometry and fancyhdr packages are very adaptable. These are the defaults I would like you to use.
\usepackage[a4paper,top=2.5cm,bottom=2.5cm,left=2.75cm,right=2.75cm,marginparwidth=2.25cm,includeheadfoot,twoside]{geometry}
\usepackage{fancyhdr}
\pagestyle{fancy}
\headheight=18pt
\footskip=18pt
\fancyhf{}
% NOTE: This sets things up so that there is a header with the short version of the title on the left, and the page number on the right
\fancyhead[L]{\bfseries \theShortTitle}
\fancyhead[R]{\bfseries \thepage}
\renewcommand{\headrulewidth}{1pt} 

% Bibliography style
\usepackage{cite}                  % Useful for sorting and compressing citations
\bibliographystyle{abbrv}

% Line spacing
\usepackage[skip=0.8\baselineskip,indent=0pt]{parskip} % If you have an old download of LaTeX this might cause problems.


% AMS Packages for maths typesetting
% NOTE: These are extremely useful!!! They are so useful, I am even leaving this in the barebones version.
\usepackage{amsmath}
\usepackage{amsfonts}
\usepackage{amssymb}

% Importing graphics and defining the graphics path
% NOTE: The second part of this means you can put graphics in a subfolder called Figures and LaTeX will find them.
\usepackage{graphicx}
\graphicspath{{./}{./Figures/}}

% TikZ packages and useful libraries
% NOTE: TikZ and pgfplots are great for drawing your own figures and plots in LaTeX (see TikZ/PGF and PGFplots manuals online for details)
% I have removed the TikZ section from the barebones preamble

% Miscellaneous packages
% NOTE: Don't be afraid to include more packages (e.g. booktabs for pretty tables) or remove packages you don't use.
% I have removed all of the packages excelt Babel from the barebones preamble.
\usepackage[english]{babel}       % Hyphenation etc. for English words

% User defined commands
% NOTE: The fact that you can define your own commands can be very useful. I have removed these from the barebones version

% Hyperlinking within text.
% NOTE: This must always be the last part of the preamble (before the doucment)
\usepackage{hyperref}
\hypersetup{
	unicode=true,                 % non-Latin characters in Acrobat bookmarks
	pdftoolbar=true,              % show Acrobat toolbar?
	pdfmenubar=true,              % show Acrobat menu?
	pdffitwindow=true,            % page fit to window when opened
	pdftitle={\theShortTitle},    % title of pdf document
	pdfauthor={\theAuthors},      % author of pdf document
	pdfsubject={},                % subject of the document
	pdfnewwindow=true,            % links in new window
	pdfkeywords={},               % list of keywords
	colorlinks=true,              % false: boxed links; true: colored links
	linkcolor=black,              % color of internal links
	citecolor=black,              % color of links to bibliography
	filecolor=black,              % color of file links
	urlcolor=blue                 % color of external links
}



\begin{document}

% Create the title and make sure there isn't a page number at the bottom of the page
\maketitle
\thispagestyle{empty}	

\begin{abstract}
	
	Brief description of a thing.
	
\end{abstract}


% Reset page numbering so document starts on page 1
\newpage
\setcounter{page}{1}

\section{Introduction}

We are going to do a thing because reasons.

\section{Methods}

This is how we will do the thing.

\section{Results}

This is what happened when did the thing.

\section{Discussion}

Here are some thoughts on what doing the thing means and about other things we want to do now.

\section{Conclusion}

We did a thing that is very meaningful because reasons.


\newpage
\bibliography{example-bib}


\newpage
\appendix

\section{Here is an appendix}
\label{sec:Appendix}

Here are some extra details about things.

\end{document}
